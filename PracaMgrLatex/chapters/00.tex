\subsubsection*{Tytuł pracy} 
\Title

\subsubsection*{Streszczenie}  
Analiza algorytmów wyszukujących zawartość tekstową w ASCII, sprawdzenie 
szybkości działania, zużycia zasobów oraz ich wydajności. Algorytmy zostały 
porównane w języku Golang, który jest zaopatrzony w wiele narzędzi testujących.
Porównano wydajność programu wykorzystującego najszybszą implementacje.

\subsubsection*{Słowa kluczowe} 
Algorytmy, Linux, Wyszukiwanie, Wydajność, Optymalizacja

\subsubsection*{Thesis title} 
\begin{otherlanguage}{british}
\TitleAlt
\end{otherlanguage}

\subsubsection*{Abstract} 
\begin{otherlanguage}{british}
Analysis of algorithms for searching textual content in ASCII, checking the 
speed of operation, resource consumption, and their efficiency. The algorithms 
were compared in the Golang language, which is equipped with many testing tools.
The performance of the program utilizing the fastest implementation was compared.

\end{otherlanguage}
\subsubsection*{Key words}  
\begin{otherlanguage}{british}
Algorithms, Linux, Searching, Performance, Optimization
\end{otherlanguage}

1. porównanie słów dla 3 algorytmów:
wan, main, window, analysis, informatyka, book desc (no case)
Wykorzystanie aplikacji grep, rg, i porównanie ich na archiwach
zestawienia pomiedzy programami
2. wykresy w odniesieniu do nazwy słowa (jeszcze raz, z dłuższymi opisami co jest testowane)
3. zakres - konstrukcja bazy, metody zapisu, porówaninie na archiwum
4. podsumowanie - kolejnym rozbudowaniem byłoby pokazanie miejsca w którym wyszukiwanie wystąpiło

misc:
kodowanie znaków jest inne w docx (nie robimy)


