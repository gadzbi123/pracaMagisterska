
\chapter{Podsumowanie}

%\begin{itemize}
%\item Jaki problem rozwiązałæm?
%\item Jak ten problem rozwiązałæm?
%\item Jakie są dobre i słabe strony mojego rozwiązania?
%\item Czy mogę sformułować jakieś rekomendacje?
%\end{itemize}

% \begin{itemize}
% \item syntetyczny opis wykonanych prac
% \item wnioski
% \item możliwość rozwoju, kontynuacji prac, potencjalne nowe kierunki
% \item Czy cel pracy zrealizowany? 
% \end{itemize}

\section{Opis wykonanych prac}

W pierwszej części przeprowadzono badania na algorytmach, porównując szybkość
ich wykonania na zbiorze plików. Dowiedziono wtedy, że na czasy wykonania 
znacznie wpływa czas alokowania nowych zasobów przez program. Udało się dowieść,
że po wykorzystaniu tego samego bufora (\textbf{BWP}) do kolejnych przeszukań, przyspieszono 
algorytm BM o około $20 \%$. Wynika to z różnicy pomiędzy
wykresami \ref{fig:GraphFirstAttempt} oraz \ref{fig:GraphPreAllocBM}.

Zaletą takie rozwiązania jest to, że bufor pozwala na zmniejszenie czasów
na alokowanie pamięci pomiędzy wykonaniami. \textbf{BWP} zasadniczo nie posiada wad w
tej implementacji. 

Ponad udało się pokazać, że stworzenie jednego bufora na przechowywanie treści
pliku redukuje dodatkowo czas pomiędzy wykonaniami o około $8 \%$, co stanowi
zaletę tego rozwiązania. Można to zaobserwować porównując różnice czasów wykonań
w \ref{fig:GraphPreAllocBM} i \ref{fig:GraphStaticPreallocAndFileBuffer}. Wadą
tego rozwiązania jest to, że rozmiar buforu na plik musi być zanany przed
wykonaniem programu. Ostatecznie dowiedziono, że algorytm Boyera-Moore'a będzie
najlepszym wyborem dla danego zbioru danych (tab. \ref{tabela:typyMIMEdataset}).

Następnie przeprowadzono badania na zbiorze archiwów, wykorzystując najszybszy algorytm 
Boyera-Moore'a. Porównano działanie programów \textbf{ugrep}, \textbf{zgrep},
\textbf{ripgrep} oraz autorskiego \textbf{gsearch}. Z badania wynikło, że
jedynie narzędzie \textbf{ripgrep} oraz \textbf{gsearch} mogą być mierzone pod
względem ilości wyszukiwań i prędkości. Narzędzia \textbf{ugrep} oraz 
\textbf{zgrep} nie znalazły żadnych wyników. 

Zaletą programu \textbf{ripgrep} okazała się szybkość wykonania w przypadku, gdy
pliki znajdują się w pamięci tymczasowej, co wynika z rysunku 
\ref{fig:wykresPorównaniaCzasówWyszukań}. Zaletą programu \textbf{gsearch} jest
jego szybkość wykonania, kiedy pliki zbioru są czytane po raz pierwszy. Wynika
to z rysunku \ref{fig:wykresPorównaniaCzasówWyszukańUncached}. 

Wadą narzędzia \textbf{ripgrep} okazało się wolniejsze wykonywanie programu po raz pierwszy.
Wadą \textbf{gsearch} było wolniejsze wykonywanie kolejnego identycznego wyszukiwania.

W ostatniej części badań odkryto, że czas wykonania algorytmu zajmuje jedynie
$0,3 \%$ całego wykonania programu \textbf{gsearch} (rys. \ref{fig:profilerGsearch1}).
Z obrazka wynika, że funkcja main wykonywała się 21,57 s, a sam algorytm 
Boyera-Moore'a jedynie 0,72 s. Zdecydowaną większość czasu oczekiwano na 
system operacyjny, aby dostarczył zawartości plików w funkcji 'cgocall'. 
(rys. \ref{fig:profilerGsearch1}). Wadą wykonania ekstrakcji archiwów w funkcji 
'cgocall' jest dłuższy czas wykonania. Zaletą jest fakt, że otrzymuje się 
dokładne miejsce wystąpienia frazy w zbiorze danych.

\section{Wnioski}

Przeprowadzone badania koncentrowały się na analizie wydajności algorytmów 
wyszukujących w kontekście przeszukiwania zawartości tekstowej w różnych formatach 
plików, ze szczególnym uwzględnieniem archiwów. Metodyka badawcza obejmowała porównanie
prędkości wykonania algorytmów na mniejszym zbiorze rozpakowanych archiwów oraz 
analizę porównawczą narzędzi pod względem liczby i prędkości wyszukiwań na 
nierozpakowanym zbiorze. Testy przeprowadzono na środowisku Linux.
 
Zbiór danych testowych składał się z różnorodnych formatów plików o łącznym 
rozmiarze około 15 GB. W trakcie badań
uwzględniono problematykę różnych formatów plików, co pozwoliło na kompleksową
ocenę wydajności algorytmów w rzeczywistych warunkach. W ramach analizy
algorytmów porównano implementacje Morrisa-Pratta, Knutha-Morrisa-Pratta i Boyera-Moore'a.
Szczególną uwagę poświęcono wpływowi optymalizacji bufora wstępnego przetwarzania
na wydajność oraz znaczeniu ponownego wykorzystania buforów w kolejnych wyszukiwaniach.
Badania wykazały istotny wpływ tych optymalizacji na ogólną wydajność wyszukiwania.
Algorytm Boyera-Moore'a otrzymywał rezultaty najszybciej, dlatego że zachowywał więcej 
informacji o łańcuchu szukanym. To pokrywa się z hipotezą badań.

Istotnym elementem badań było porównanie autorskiego narzędzia \textbf{gsearch} z 
popularnymi rozwiązaniami takimi jak \textbf{ugrep}, \textbf{zgrep} i 
\textbf{ripgrep}. Analiza objęła 
różnice działania programów, w kontekście różnych archiwów oraz wpływ pamięci
podręcznej na wydajność wyszukiwania. 

Wykazano, że operacje I/O, w tym odczyt z dysku i ekstrakcja archiwów, stanowią największe
obciążenie, podczas gdy sam algorytm wyszukiwania zajmuje relatywnie małą część
całkowitego czasu wykonania. Autorskie rozwiązanie \textbf{gsearch}, choć ustępuje
wydajnością \textbf{ripgrep} w przypadku kolejnych wyszukiwań, oferuje unikalną 
funkcjonalność znalezienia wyników w złożonej strukturze zagnieżdżonych archiwów.
Funkcja \textbf{ripgrep} pozwala znaleźć wystąpienia w archiwum bez dokładnej
lokalizacji ścieżki w archiwum.

\section{Możliwości rozwoju}

Program można usprawnić w celu wyszukiwania większej ilości zawartości przy
pomocy dodatkowej implementacji dla transkrypcji. Pozwoliłaby ona na wyszukanie
treści w plikach audio, ponieważ istnieją darmowe narzędzia pozwalające na taką
konwersję.

Kolejnym elementem, który można rozważyć w celu kontynuacji pracy, byłoby 
wykorzystanie OCR (ang. \english{Optical Character Recognition}). Zaimplementowanie
takiego rozwiązania pozwoli pozyskać treść ze zdjęć oraz plików pdf, które składają 
się ze zdjęć i tekstowych skanów treści.

W celu uzyskania lepszych rezultatów można, zamiast wykorzystania gotowej 
biblioteki, stworzyć autorską bibliotekę dekompresującą. Obejmowałaby ona 
brakujące i nie poprawnie działania niektórych formatów archiwów. To zadanie
wymaga ustalenia granicy, jak bardzo archiwum może być uszkodzone, żeby można
było odczytać z niego dane.

Kolejnym usprawnieniem dla programu byłoby wprowadzenie pamięci podręcznej (ang. \english{caching}).
Takie rozwiązanie pozwoliłoby na zapamiętanie plików, które już kiedyś dekompresowano.
Należy jednak przechować informacje o tym, czy suma kontrolna (ang. \english{hash})
archiwum nie zmieniła się, pomiędzy wykonaniami programu.

Dodatkowo można pozwolić programowi działać na większej ilości wątków. To 
wymaga dodatkowej synchronizacji pomiędzy odczytem archiwów, natomiast
Golang jest bardzo łatwym językiem do wprowadzania takich zmian.

Kolejnym kierunkiem do rozwoju jest wykorzystanie wyrażeń regularnych podczas
wyszukiwania zawartości. Taka implementacja na pewno spowolni algorytm 
wyszukujący, natomiast pozwoli na uzyskanie większej ilości rezultatów.