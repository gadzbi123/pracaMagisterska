
\chapter{Podsumowanie}

%\begin{itemize}
%\item Jaki problem rozwiązałæm?
%\item Jak ten problem rozwiązałæm?
%\item Jakie są dobre i słabe strony mojego rozwiązania?
%\item Czy mogę sformułować jakieś rekomendacje?
%\end{itemize}

\begin{itemize}
\item syntetyczny opis wykonanych prac
\item wnioski
\item możliwość rozwoju, kontynuacji prac, potencjalne nowe kierunki
\item Czy cel pracy zrealizowany? 
\end{itemize}

\section{Możliwości rozwoju}

Program można usprawnić w celu wyszukiwania większej ilości zawartości przy
pomocy dodatkowej implementacji dla transkrypcji. Pozwoliłaby ona na wyszukanie
treści w plikach audio, a istnieje wiele narzędzi, które są darmowe i już
na to pozwalają.

Kolejnym elementem, który można rozważyć w celu kontynuacji pracy, byłoby 
wykorzystanie OCR (ang. \english{Optical Character Recognition}). Zaimplementowanie
takiego rozwiązania pozwoli pozyskać treść ze zdjęć oraz pdfów, które składają 
się ze zdjęć i tekstowych skanów treści.

W celu uzyskania lepszych rezultatów można, zamiast wykorzystywać gotową 
bibliotekę — stworzyć bibliotekę dekompresującą, wszystkich brakujących i nie
poprawnie działających formatów archiwów. Jest to jednak dość wymagające zadanie
oraz możliwość testowania tego rozwiązania jest ograniczona. Jak bardzo 
plik może być uszkodzony, żeby można było odczytać z niego dane, które nie są
niepoprawne.

Kolejnym usprawnieniem dla programu byłoby wprowadzenie pamięci podręcznej (ang. \english{caching}).
Takie rozwiązanie pozwoliłoby na zapamiętanie plików, które już kiedyś dekompresowano.
Pliki skompresowane zwykle nie mogą być zapisane, wiec po pierwszej dekompresji
można zachować tylko znalezione wystąpienia. Należy jednak przechować 
informacje o tym, czy suma kontrolna (ang. \english{hash}) archiwum się nie 
zmieniła.