\chapter{Wstęp}

\section{Wprowadzenie do problemu}
Analiza struktur danych o dużych rozmiarach, szczególnie gdy mamy do czynienia
z rozproszoną strukturą katalogów, co stanowi istotne wyzwanie w dziedzinie 
inżynierii oprogramowania i zarządzania danymi. 

Jednym ze sposobów zachowywania danych i zmniejszenia ich objętości jest 
archiwizacja plików. Takie rozwiązanie jest bardzo przydatne w przypadku chęci
zmniejszenia ilości danych przechowywanych, a także w przypadku chęci 
dystrybucji danych dla innych użytkowników jak zostało to zrobione, gdy 
repozytorium danych zostało przekazane do analizy w celu wykonania tejże pracy.

W kwestii technicznej należało rozważyć sposób efektywnego zarządzania pamięcią
w przypadku czytania dużej ilości danych z dysku, opóźnienia związane z 
wydajnością operacji I/O, które należało ograniczyć do minimum.

Problem wyszukiwania danych nastąpił w momencie wyszukiwania dużej ilość 
zawartości. Posiadane archiwum wynosi 14,7 GB danych, niektóre z plików są 
zarchiwizowane, co utrudnia odczytanie z nich danych. 

Implementacja poznanych algorytmów pozwoliła na określenie, który algorytm 
optymalnie wyszukuje zawartość w wykorzystywanym zbiorze danych. A niewielkie
różnice sposobu odczytu danych wpływały na prędkość wydajność wyszukania.

Praca będzie obejmowała analizę algorytmów oraz różnice implementacji wpływające
na prędkość. 

\section{Cel Pracy}
Głównym celem pracy jest dogłębna analiza i porównanie algorytmów wyszukujących zawartość tekstową w formacie ASCII.
Badanie ma na celu określenie, który z algorytmów charakteryzuje się najwyższą szybkością działania,
optymalnym zużyciem zasobów oraz największą wydajnością. Poprzez systematyczne testowanie i pomiary,
praca dąży do identyfikacji najbardziej efektywnego rozwiązania w kontekście wyszukiwania tekstowego.

Istotnym aspektem celu pracy jest również zbadanie możliwości języka programowania Golang
w kontekście implementacji i testowania algorytmów wyszukiwania.
Golang, znany ze swojej wydajności i bogatego zestawu narzędzi testujących,
stanowi idealną platformę do przeprowadzenia kompleksowej analizy porównawczej.
Wykorzystanie tego języka pozwoli na rzetelną ocenę wydajności poszczególnych 
algorytmów oraz zidentyfikowanie optymalnego rozwiązania.

Ponadto, celem pracy jest dostarczenie rekomendacji dotyczących wyboru najlepszego algorytmu wyszukiwania
tekstowego w ASCII dla danego zbioru danych. Wyniki analizy mają służyć jako źródło informacji dla 
programistów i badaczy, umożliwiając im podejmowanie świadomych decyzji przy 
implementacji systemów wyszukiwania tekstowego. Praca ma na celu wniesienie 
wkładu w rozwój wiedzy na temat efektywności algorytmów i optymalizacji procesów wyszukiwania, 
jak również procesów optymalizacyjnych wykorzystywanych w nowoczesnych językach programowania. 

\section{Zakres Pracy}
Zakres pracy obejmuje szczegółową analizę i porównanie wybranych algorytmów wyszukiwania zawartości tekstowej w formacie ASCII.
Badaniu zostaną poddane różne podejścia i techniki wyszukiwania, takie jak algorytmy oparte na dopasowywaniu wzorców,
indeksowaniu czy strukturach danych. Praca skupi się na dogłębnym zrozumieniu
działania poszczególnych algorytmów oraz ich implementacji w języku Golang.

W ramach zakresu pracy przeprowadzone zostaną kompleksowe testy wydajnościowe, mierzące szybkość działania,
zużycie zasobów oraz efektywność poszczególnych algorytmów. Wyniki pomiarów zostaną poddane analizie statystycznej
w celu identyfikacji istotnych różnic w wydajności między algorytmami.

Ważnym elementem zakresu pracy będzie również analiza złożoności obliczeniowej 
oraz wpływu różnych czynników, takich jak długość wyszukiwanego wzorca czy rozmiar
przeszukiwanego tekstu, na wydajność algorytmów. Pozwoli to na zrozumienie ograniczeń
i mocnych stron poszczególnych podejść oraz identyfikację optymalnych warunków ich stosowania.

Dodatkowo praca zawiera propozycje dalszych badań i kierunków rozwoju w dziedzinie algorytmów wyszukiwania
tekstowego, wskazując potencjalne obszary usprawnień i innowacji.

% \textbf{TODO opis ogólny rozdziałów} 
% \begin{itemize}
% \item wprowadzenie w problem/zagadnienie 
%\item osadzenie problemu w dziedzinie 
%\item cel pracy 
%\item zakres pracy 
%\item zwięzła charakterystyka rozdziałów 
% \end{itemize}

