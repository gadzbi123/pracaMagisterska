\chapter{Wstęp}

\section{Wprowadzenie do problemu}
Analiza przeszukiwania struktur danych o dużych rozmiarach z wieloma typami danych stanowi
istotne wyzwanie w dziedzinie inżynierii oprogramowania i zarządzania danymi. 

Jednym ze sposobów zachowywania danych i zmniejszenia ich objętości jest 
archiwizacja plików. Takie rozwiązanie jest bardzo przydatne w przypadku chęci
zmniejszenia ilości danych przechowywanych, a także w dystrybucji danych dla
innych użytkowników. Otrzymana biblioteka danych była przekazana do analizy w 
celu wykonania tejże pracy.

W kwestii technicznej należało rozważyć sposób efektywnego zarządzania pamięcią
w przypadku czytania dużej ilości danych z dysku, jak i opóźnienia związane z 
wydajnością operacji I/O.

Problem wyszukiwania danych nastąpił w momencie wyszukiwania dużej ilość 
zawartości. Oryginalne archiwum wynosi 15 GB danych, gdzie pliki są 
zarchiwizowane, co utrudnia odczytanie z nich danych. 

Wykonanie operacji odczytu archiwów będzie miało kluczowe znaczenie w 
otrzymaniu informacje o miejscu znajdowania się treści, choć może znacznie 
wpłynąć na wydajność programu.

Implementacja poznanych algorytmów pozwoliła na określenie, który algorytm 
optymalnie wyszukuje zawartość w wykorzystywanym zbiorze danych. A niewielkie
różnice sposobu odczytu danych wpływały na prędkość wydajność wyszukania.

Dodatkowo odczytywanie archiwów w archiwach wymaga kilkukrotnego wyodrębnienia
danych. Aby otrzymać odpowiednią ilość wyników, należy wyodrębnić 
wszystkie zagnieżdżone archiwa. Zasadniczo archiwa po wydobyciu,
tworzą drzewo plików, w którym mogą znajdować się kolejne archiwa, z kolejnym drzewem
plików itd. Odczytanie zawartości odbędzie się poprzez przejście po drzewie 
każdego z wyodrębnionego archiwum.

\section{Cel Pracy}

Celem niniejszej pracy jest analiza algorytmów wyszukujących zawartość tekstową 
w standardzie ASCII oraz ocena ich efektywności w systemie operacyjnym Linux. 
Kluczowym zagadnieniem badawczym jest porównanie różnych metod przeszukiwania 
tekstu pod względem szybkości działania oraz dokładności. Skupiono się na 
implementacjach tych algorytmów w języku programowania Golang, który oferuje
rozbudowane narzędzia testujące i profilujące kod.

Praca ma na celu nie tylko teoretyczne zestawienie wybranych algorytmów, ale 
także ich praktyczną implementację i porównanie w rzeczywistym środowisku 
obliczeniowym. W tym kontekście istotnym aspektem jest zbadanie wydajności 
różnych podejść do przeszukiwania tekstu w dużych zbiorach danych, w tym w 
archiwach z zagnieżdżonymi folderami.

Kolejnym celem jest ocena poprawności działania zaimplementowanych algorytmów, 
co oznacza sprawdzenie, czy znajdują one wszystkie wystąpienia szukanych fraz w 
sposób zgodny z oczekiwaniami. W szczególności badane będą przypadki graniczne, 
takie jak wyszukiwanie w dużych plikach, przeszukiwanie czy analiza wpływu 
długości wzorca na czas wyszukiwania.

Ostatecznym celem pracy jest dostarczenie rekomendacji dotyczących wyboru 
odpowiedniego algorytmu wyszukiwania tekstowego w zależności od specyfiki 
zadania. Analiza porównawcza umożliwi wskazanie rozwiązań optymalnych pod 
względem wydajności oraz zastosowania w różnych warunkach systemowych.

\section{Zakres Pracy}
Zakres pracy obejmuje szczegółową analizę algorytmów wyszukiwania tekstu w Golang.
W pierwszej części pracy przedstawione zostaną teoretyczne podstawy algorytmów wyszukiwania,
w tym klasyczne podejścia, takie jak algorytm Knutha-Morrisa-Pratta, Boyera-
Moore'a oraz inne techniki wyszukiwania w tekście ASCII.

Kolejnym etapem będzie implementacja wybranych algorytmów w języku Golang oraz 
ich optymalizacja pod kątem wydajności. Przeprowadzone zostaną testy porównawcze, 
w których oceniana będzie szybkość wyszukiwania oraz skuteczność w znajdowaniu
wzorców tekstowych. Dodatkowo uwzględniona zostanie analiza wpływu wielkości
pliku oraz długości wzorca na efektywność działania poszczególnych metod.

Praca obejmuje również testowanie wydajności algorytmów w rzeczywistym 
środowisku systemu Linux. Badania będą prowadzone w konsolowym interfejsie 
użytkownika, gdzie zaimplementowane algorytmy zostaną przetestowane na 
rzeczywistych zbiorach archiwalnych. Zostaną również porównane czasy wykonania
wyszukiwania w zależności od długości frazy.

Na zakończenie pracy zostaną zaprezentowane wyniki przeprowadzonych badań wraz 
z wnioskami dotyczącymi efektywności analizowanych algorytmów. Na podstawie 
uzyskanych wyników zostaną sformułowane rekomendacje dotyczące stosowania 
poszczególnych metod wyszukiwania tekstowego w zależności od rodzaju danych 
oraz wymagań wydajnościowych.

% \textbf{TODO opis ogólny rozdziałów} 
% \begin{itemize}
% \item wprowadzenie w problem/zagadnienie 
%\item osadzenie problemu w dziedzinie 
%\item cel pracy 
%\item zakres pracy 
%\item zwięzła charakterystyka rozdziałów 
% \end{itemize}

