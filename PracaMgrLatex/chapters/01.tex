\chapter{Wstęp}

\section{Wprowadzenie do problemu}
Analiza struktur danych o dużych rozmiarach, szczególnie gdy mamy do czynienia
z rozproszoną strukturą katalogów, co stanowi istotne wyzwanie w dziedzinie 
inżynierii oprogramowania i zarządzania danymi. 

Jednym ze sposobów zachowywania danych i zmniejszenia ich objętości jest 
archiwizacja plików. Takie rozwiązanie jest bardzo przydatne w przypadku chęci
zmniejszenia ilości danych przechowywanych, a także w przypadku chęci 
dystrybucji danych dla innych użytkowników jak zostało to zrobione, gdy 
repozytorium danych zostało przekazane do analizy w celu wykonania tejże pracy.

W kwestii technicznej należało rozważyć sposób efektywnego zarządzania pamięcią
w przypadku czytania dużej ilości danych z dysku, opóźnienia związane z 
wydajnością operacji I/O, które należało ograniczyć do minimum.

Problem wyszukiwania danych nastąpił w momencie wyszukiwania dużej ilość 
zawartości. Posiadane archiwum wynosi 14.7 GB danych, niektóre z plików są 
zarchiwizowane co utrudnia odczytanie z nich danych. 

Nie mniej jednak, posiadane narzędzia w systemach dają dużą dowolność w 
wyszukania zawartości. Istnieje również możliwość napisania własnych 
implementacji, które mogą zostać zoptymalizowane do danych, które odczytujemy.

Praca będzie obejmowała analizę algorytmów jak również analizę porównawczą 
narzędzi stosowanych do wyszukiwania tekstu w podobny sposób. 
\textbf{TODO opis ogólny rozdziałów} 
% \begin{itemize}
% \item wprowadzenie w problem/zagadnienie 
%\item osadzenie problemu w dziedzinie 
%\item cel pracy 
%\item zakres pracy 
%\item zwięzła charakterystyka rozdziałów 
% \end{itemize}

